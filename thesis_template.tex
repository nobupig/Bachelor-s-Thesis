% ========== 日本語卒論テンプレート ==========
\documentclass[a4paper,12pt]{bxjsreport}

% -------------------------------------
% ★★ 日本語表示の必須設定(最重要) ★★
% -------------------------------------
\usepackage{luatexja}
\usepackage{luatexja-fontspec}

% 使う日本語フォントを指定(TexLive標準で必ず存在)
\setmainjfont{IPAexMincho}      % 明朝体
\setsansjfont{IPAexGothic}      % ゴシック
\setmonojfont{IPAexGothic}

% ----- 必要なパッケージ -----
\usepackage{amsmath, amssymb}
\usepackage{graphicx}
\usepackage{here}
\usepackage{booktabs}
\usepackage{longtable}
\usepackage{caption}
\usepackage{subcaption}
\usepackage{hyperref}
\usepackage{setspace}
\usepackage{geometry}
\usepackage[normalem]{ulem}  

% ----- 余白設定 -----
\geometry{
  left=25mm,
  right=25mm,
  top=25mm,
  bottom=25mm
}

% ----- 行間 -----
\setstretch{1.25}

\begin{document}

%==============================
%  表紙(固定デザイン)
%==============================

\begin{titlepage}
\centering

\vspace*{15mm}

% 校章(任意:画像ファイルを thesis フォルダに置く場合)
% \includegraphics[width=25mm]{school_logo.png}
% \vspace{10mm}

{\Large 令和 7 年度 \quad 卒業研究論文}\\[20mm]

% タイトル(中央寄せ・2行)
{\Large \textbf{手指に付着したウイルス量の測定と}}\\[3mm]
{\Large \textbf{洗浄後の残存リスク評価に関する実験的研究}}\\[20mm]

% 学生一覧
{\large
\begin{tabular}{r l}
92100104 & 上田 涼介 \\
92100124 & 岡本 悠椰 \\
92100108 & 川瀬 羽駆 \\
92005010 & 蔵岡 倫哉 \\
92000130 & 押井 勇希 \\
92100113 & 寺地 翔  \\
92005035 & 森山 大史 \\
\end{tabular}
}\\[20mm]

% 所属
{\large 近畿大学工業高等専門学校}\\[3mm]
{\large 総合システム工学 都市環境コース(土木系)}\\[10mm]

% 指導教員
{\large 指導教員:安井宣仁 准教授}

\vfill

\end{titlepage}

\clearpage

% ===========================
% 目次
% ===========================
\tableofcontents
\clearpage

% ===========================
% 第1章のみ(暫定)
% ===========================
\chapter{序論}
1.研究の背景~

\textbf{水系感染症の世界情勢~}

安全な飲料水や衛生設備が十分に整っていない地域では、細菌による水系感染症が多発している。そのため、それらの地域では安全で清潔な水へのアクセスが非常に困難であり、健康被害が深刻化している。一方で、日本は水道の普及率が98.3%と非常に高く、ほとんどの地域で清潔で安全な水を安定して利用することができる。この点からも、日本は他国と比較して水資源の確保や衛生面で大きく進んでいることがわかる。しかし、世界全体で見ると、いまだに多くの人々が安全で清潔な水を十分に利用できない状況にある。汚染された水を飲んだり、生活用水として使用したりすることで感染症にかかる人が後を絶たず、特に発展途上国では深刻な社会問題となっている。代表的な水系感染症としては、コレラ、赤痢、チフス、A型肝炎などがあり、これらはいずれも汚れた水や不衛生な環境が主な原因で発生する。世界保健機関(WHO)の報告によると、安全な飲料水を利用できない人は世界で約20億人以上にのぼるとされており、安全な水の確保は依然として大きな課題である。このような背景から、本研究では、災害時や水道が利用できず安全な水へのアクセスが困難な状況を想定し、代替水源として微生物を含む可能性のある水を使用した場合の衛生面について調査を行う。特に、手洗いやトイレの使用など、日常的な行動において微生物が掌や指先にどの程度付着するのかを測定し、感染のリスクや衛生対策の重要性を明らかにしていくことを目的とする。~

\begin{enumerate}
\def\labelenumi{\arabic{enumi}.}
\setcounter{enumi}{1}
\item
  既存研究の限界~
\end{enumerate}

~

病院エレベーターの押しボタンと手指を対象とした菌の伝播に関する一考察

\href{https://www.jstage.jst.go.jp/search/global/_search/-char/ja?item=8&word=\%E6\%9D\%BE\%E9\%87\%8E+\%E5\%AE\%B9\%E5\%AD\%90\%22\%20/o\%20\%22\%E6\%9D\%BE\%E9\%87\%8E\%20\%E5\%AE\%B9\%E5\%AD\%90,\%20\%E5\%B1\%B1\%E5\%8F\%A3\%E5\%A4\%A7\%E5\%AD\%A6\%E9\%99\%84\%E5\%B1\%9E\%E7\%97\%85\%E9\%99\%A2\%E6\%A4\%9C\%E6\%9F\%BB\%E9\%83\%A8}{松野
容子},\,\href{https://www.jstage.jst.go.jp/search/global/_search/-char/ja?item=8&word=\%E6\%B0\%B4\%E9\%87\%8E+\%E7\%A7\%80\%E4\%B8\%80}{水野
秀一},\,\href{https://www.jstage.jst.go.jp/search/global/_search/-char/ja?item=8&word=\%E5\%A4\%A7\%E5\%BE\%B3+\%E5\%84\%AA\%E5\%AD\%90}{大徳
優子}~

病院環境における菌の分布やその動態の一端を明らかにする目的で,
病棟エレベーターの押しボタンと手指を対象とする細菌学的検討を行った.使用人数別
(1~8人) および経時間的 (30~120分) に押しボタン上から検出された菌数は,
おおむね0から最高150~300colony forming units (CFU) で,
そのうちの70\%以上が20 CFU未満であった.しかし,
時に1200CFUもの菌が検出されるなど,
一部のヒトの手指の汚染状況に応じて接触後に非常に汚染された状態が生じうることが明らかとなった~

高度汚染した手指の衛生学的手洗いの検討

\href{https://www.jstage.jst.go.jp/search/global/_search/-char/ja?item=8&word=\%E7\%9F\%A2\%E9\%87\%8E+\%E4\%B9\%85\%E5\%AD\%90}{矢野
久子},\,\href{https://www.jstage.jst.go.jp/search/global/_search/-char/ja?item=8&word=\%E5\%B0\%8F\%E6\%9E\%97+\%E5\%AF\%9B\%E4\%BC\%8A}{小林
寛伊},\,\href{https://www.jstage.jst.go.jp/search/global/_search/-char/ja?item=8&word=\%E5\%A5\%A5\%E4\%BD\%8F+\%E6\%8D\%B7\%E5\%AD\%90}{奥住
捷子}~

病院感染防止のために衛生学的手洗いは重要である. しかし,
病棟などでの手洗いの頻度, 手洗い時間の短さが指摘されている.
高度に手指汚染した場合の手洗い時間,
手洗い方法とその効果について検討した.~

~

手洗い効果の細菌学的考察~

\href{https://www.jstage.jst.go.jp/search/global/_search/-char/ja?item=8&word=\%E7\%9F\%B3\%E7\%94\%B0+\%E5\%92\%8C\%E5\%A4\%AB}{石田
和夫},\,\href{https://www.jstage.jst.go.jp/search/global/_search/-char/ja?item=8&word=\%E4\%B8\%89\%E6\%B5\%A6+\%E8\%8B\%B1\%E9\%9B\%84}{三浦
英雄}~

1997〜1999年の3年間にわたり, 本学学生399人を対象に水洗い,~セッケン,
0.2\%塩化べンザルコニウム30秒間及び数秒間浸漬, 70\%エタノール,
薬用Mセッケンの6通りの方法で手指を洗浄または殺菌消毒させ,~手洗い前と手洗い後の手指の生菌数の比較を行った.~

~

高齢者介護施設における入居者の手指衛生の実態~

\href{https://www.jstage.jst.go.jp/search/global/_search/-char/ja?item=8&word=\%E4\%B8\%AD\%E6\%9D\%91+\%E6\%98\%8E\%E4\%B8\%96}{中村
明世},\,\href{https://www.jstage.jst.go.jp/search/global/_search/-char/ja?item=8&word=\%E5\%B2\%A1\%E7\%94\%B0+\%E6\%B7\%B3\%E5\%AD\%90}{岡田
淳子},\,\href{https://www.jstage.jst.go.jp/search/global/_search/-char/ja?item=8&word=\%E9\%A3\%AF\%E7\%94\%B0+\%E5\%BF\%A0\%E8\%A1\%8C}{飯田
忠行}~

本研究の目的は,
高齢者介護施設入居者の手指衛生の実態を把握することである.
入居者をケアする職員240名を対象に入居者の手指衛生の実施頻度を調査した.
入居者は活動状況別に3群に分類し,
手指衛生の方法と実施場面の違いによる手指衛生の実施頻度を調査し,
3群間で比較した.
入居者の活動状況によって介護現場で実施されている手指衛生の方法は異なり,
寝たきりのベッド上群がウェットティッシュ,
排泄をポータブルトイレで行うポータブルトイレ群はアルコール,
ひとりで移動できる見守り群は流水であった. 手指衛生の場面は,
ポータブルトイレ群と見守り群の排尿後および排便後の実施頻度が高い傾向にあった.\,~

~~

3. 本研究の目的と重要性~

現在までに発表されている実験データの中には、人が日常生活の中でさまざまな物体に手で触れた際、どの程度の微生物が皮膚表面に付着するかを測定した研究がいくつか存在している。これらの研究では、主に日常的な生活環境や医療現場など、比較的衛生状態が管理された状況を想定しているものが多い。しかし、災害発生時のように、衛生環境が急激に悪化し、安全な水の供給が途絶えるような特殊な状況を対象とした研究は、現時点ではほとんど見られないのが現状である。そこで本研究では、災害時において水道などの安全な水源の利用が困難となり、やむを得ず河川水や雨水、貯留水などの代替水源を使用せざるを得ない場合を想定し、手指にどの程度の微生物が付着・残存するのかを詳細に測定することを目的とする。特に、手洗いの方法や使用する水の種類、接触時間などの実験条件を変化させることで、微生物付着量の違いを定量的に明らかにすることを目指す。また、得られた測定データをもとに統計的・科学的な分析を行い、災害時における感染症発生のリスクを数値的に評価する。さらに、その結果を踏まえて、災害時や安全な水資源へのアクセスが制限される状況でも実践可能な効果的な手指衛生の方法や、感染症予防に有効な衛生対策を検討・提案することを本研究の最終的な目的とする。~

~

~


% ===========================
% 以下の章はまだコメントでOK
% ===========================

\chapter{掌面積および付着水量評価手法の確立}
 2.1:掌面積・体積の測定手法

・概要\ldots 本研究では、災害時における河川水などから掌につく病原微生物の付着量、ウイルスの付着量を調べることを目的とし、各個人の掌の付着量を正確に測るために、様々な器具・機械を使用した。

実験材料

\begin{quote}
掌の面積・体積を測定するために以下の器具を使用した。
\end{quote}

(使用器具)

 ・ノギス\ldots 掌の厚みを測定するため。

\begin{quote}
・フラットベットスキャナー(CanonCanoScan)\ldots 掌をスキャンするため
\end{quote}

・定規(10cm以上の物)\ldots 掌の幅を正確に測るため

 ・画像処理ソフト(ImageJ)\ldots 掌面積を求めるため

実験方法と条件

  ・被験者9名を対象として掌面積の測定を行った。各被験者について、左右の手それぞれ3回ずつスキャナーで測定を行い、計1人あたり6セットのデータを取得した。

   掌の測定方法の手順は以下の通りである。

面積測定方法

\textbf{手順}

\begin{quote}
① 掌をスキャン\\
・掌をスキャナーのガラス面に軽くのせる。掌を軽く広げ、掌全体が写るようにする。\\
・定規を手の横に置いて同時にスキャンする。定規が「実際の」スケール(cm)を示す基準になる。

② 画像データの保存\\
・スキャン画像をパソコンに保存する。\\
・画像の解像度は300~600dpi程度が望ましい。

③ ImageJでスケール設定\\
・ImageJを起動し、スキャンした画像を開く。\\
・定規の5cm部分を「直線ツール」で選択し、[Analyze]→[Set
Scale]をクリックする。\\
・``Known distance''に{[}1.0{]}{]}、``Unit of
length''に{[}cm{]}を入力してOKを押す。

④ 掌の面積を測定\\
・「Polygon Selections」または「Freehand
Selections」ツールで掌の外周を囲む。\\
・[Analyze]→[Measure]を選択すると、掌の面積(cm²)が表示される。

⑤ ノギスによる補助測定\\
・ノギスを用いて手長(中指先端から手首まで)および手幅(親指付け根から小指付け根まで)を測

定する。\\
・得られた寸法と面積値を比較する。
\end{quote}

「注意点」

 ・境界が不明な場合は、黒マーカーや手袋で明るさや色の違いを強調するとよい。

 ・複数回スキャンし、平均値で評価すると精度が高まる。

 2.2測定結果

   掌面積の測定は、被験者について左右それぞれ3回ずつ行い、その平均値を結果として用いた。

   一人3回測定した理由としては、スキャナーの読み取りや手の置き方などによるわずかな誤差を 考慮し、複数回測定することで再現性を確認し信頼性の高いデータを得ることを目的とし行った。

   各被験者の左右の平均掌の面積および掌の体積を以下の表Xに示す。

\includegraphics[width=5.90556in,height=3.05347in]{media/media/image1.png}

2.3掌の面積および体積の左右差に関するデータ解析結果

                                             本解析では、9名の被験者における左右の掌面積(cm\textsuperscript{2})および体積(cm\textsuperscript{3})を比較し、右手と左手の間に有意な差が存在するかを検討した。対応のあるt検定(pairedt-test)を用い、有意水準は5%(p\textless0.05)とした。

  【結果】

    掌の面積では平均にわずかに大きい傾向(+1.5%)を示したが、有意差は認められなかった(p=0.0399)

    

2.3掌の面積および体積の左右差に関するデータ解析結果

                                             本解析では、9名の被験者における左右の掌面積(cm\textsuperscript{2})および体積(cm\textsuperscript{3})を比較し、右手と左手の間に有意な差が存在するかを検討した。対応のあるt検定(pairedt-test)を用い、有意水準は5%(p\textless0.05)とした。

 【結果】

   掌の面積では平均にわずかに大きい傾向(+1.5%)を示したが、有意差は認められなかった(p=0.039)  

\begin{quote}
\includegraphics[width=5.90556in,height=0.78056in]{media/media/image2.png}

【解析結果】

  掌の面積は個人差が大きく、利き手による影響は統計的に明確ではなかった。一方で掌の体積では右手が有意に大きく、利き手側の使用頻度や筋肉量の発達が反映されている可能性が高い。

  この結果は、利き手優位性(hand do
miance)の理的特徴として整合的である。

【総合的考察】

  掌面積には有意な左右差は認められなかったが、掌体積においては右手が有意に大きかった。

  今後は被験者数を増やし、利き手情報を併せて分析することで、利き手と非利き手の構造的・機能的差異をより明確に評価できると考えられる。
\end{quote}


% \chapter{掌表面の付着水分量の測定と統計解析}
% \input{chap3.tex}

% \chapter{ファージMS2付着・洗浄基礎実験}
% \input{chap4.tex}

% \chapter{洗浄条件による除去効率比較}
% \input{chap5.tex}

% \chapter{面積補正によるウイルス濃度標準化と減衰解析}
% \input{chap6.tex}

% \chapter{洗浄後の残存ウイルス量に基づくリスク評価}
% \input{chap7.tex}

% \appendix
% \chapter{付録}
% \input{appendix.tex}

\end{document}
