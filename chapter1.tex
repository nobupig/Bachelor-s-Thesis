1.研究の背景~

\textbf{水系感染症の世界情勢~}

安全な飲料水や衛生設備が十分に整っていない地域では、細菌による水系感染症が多発している。そのため、それらの地域では安全で清潔な水へのアクセスが非常に困難であり、健康被害が深刻化している。一方で、日本は水道の普及率が98.3%と非常に高く、ほとんどの地域で清潔で安全な水を安定して利用することができる。この点からも、日本は他国と比較して水資源の確保や衛生面で大きく進んでいることがわかる。しかし、世界全体で見ると、いまだに多くの人々が安全で清潔な水を十分に利用できない状況にある。汚染された水を飲んだり、生活用水として使用したりすることで感染症にかかる人が後を絶たず、特に発展途上国では深刻な社会問題となっている。代表的な水系感染症としては、コレラ、赤痢、チフス、A型肝炎などがあり、これらはいずれも汚れた水や不衛生な環境が主な原因で発生する。世界保健機関(WHO)の報告によると、安全な飲料水を利用できない人は世界で約20億人以上にのぼるとされており、安全な水の確保は依然として大きな課題である。このような背景から、本研究では、災害時や水道が利用できず安全な水へのアクセスが困難な状況を想定し、代替水源として微生物を含む可能性のある水を使用した場合の衛生面について調査を行う。特に、手洗いやトイレの使用など、日常的な行動において微生物が掌や指先にどの程度付着するのかを測定し、感染のリスクや衛生対策の重要性を明らかにしていくことを目的とする。~

\begin{enumerate}
\def\labelenumi{\arabic{enumi}.}
\setcounter{enumi}{1}
\item
  既存研究の限界~
\end{enumerate}

~

病院エレベーターの押しボタンと手指を対象とした菌の伝播に関する一考察

\href{https://www.jstage.jst.go.jp/search/global/_search/-char/ja?item=8&word=\%E6\%9D\%BE\%E9\%87\%8E+\%E5\%AE\%B9\%E5\%AD\%90\%22\%20/o\%20\%22\%E6\%9D\%BE\%E9\%87\%8E\%20\%E5\%AE\%B9\%E5\%AD\%90,\%20\%E5\%B1\%B1\%E5\%8F\%A3\%E5\%A4\%A7\%E5\%AD\%A6\%E9\%99\%84\%E5\%B1\%9E\%E7\%97\%85\%E9\%99\%A2\%E6\%A4\%9C\%E6\%9F\%BB\%E9\%83\%A8}{松野
容子},\,\href{https://www.jstage.jst.go.jp/search/global/_search/-char/ja?item=8&word=\%E6\%B0\%B4\%E9\%87\%8E+\%E7\%A7\%80\%E4\%B8\%80}{水野
秀一},\,\href{https://www.jstage.jst.go.jp/search/global/_search/-char/ja?item=8&word=\%E5\%A4\%A7\%E5\%BE\%B3+\%E5\%84\%AA\%E5\%AD\%90}{大徳
優子}~

病院環境における菌の分布やその動態の一端を明らかにする目的で,
病棟エレベーターの押しボタンと手指を対象とする細菌学的検討を行った.使用人数別
(1~8人) および経時間的 (30~120分) に押しボタン上から検出された菌数は,
おおむね0から最高150~300colony forming units (CFU) で,
そのうちの70\%以上が20 CFU未満であった.しかし,
時に1200CFUもの菌が検出されるなど,
一部のヒトの手指の汚染状況に応じて接触後に非常に汚染された状態が生じうることが明らかとなった~

高度汚染した手指の衛生学的手洗いの検討

\href{https://www.jstage.jst.go.jp/search/global/_search/-char/ja?item=8&word=\%E7\%9F\%A2\%E9\%87\%8E+\%E4\%B9\%85\%E5\%AD\%90}{矢野
久子},\,\href{https://www.jstage.jst.go.jp/search/global/_search/-char/ja?item=8&word=\%E5\%B0\%8F\%E6\%9E\%97+\%E5\%AF\%9B\%E4\%BC\%8A}{小林
寛伊},\,\href{https://www.jstage.jst.go.jp/search/global/_search/-char/ja?item=8&word=\%E5\%A5\%A5\%E4\%BD\%8F+\%E6\%8D\%B7\%E5\%AD\%90}{奥住
捷子}~

病院感染防止のために衛生学的手洗いは重要である. しかし,
病棟などでの手洗いの頻度, 手洗い時間の短さが指摘されている.
高度に手指汚染した場合の手洗い時間,
手洗い方法とその効果について検討した.~

~

手洗い効果の細菌学的考察~

\href{https://www.jstage.jst.go.jp/search/global/_search/-char/ja?item=8&word=\%E7\%9F\%B3\%E7\%94\%B0+\%E5\%92\%8C\%E5\%A4\%AB}{石田
和夫},\,\href{https://www.jstage.jst.go.jp/search/global/_search/-char/ja?item=8&word=\%E4\%B8\%89\%E6\%B5\%A6+\%E8\%8B\%B1\%E9\%9B\%84}{三浦
英雄}~

1997〜1999年の3年間にわたり, 本学学生399人を対象に水洗い,~セッケン,
0.2\%塩化べンザルコニウム30秒間及び数秒間浸漬, 70\%エタノール,
薬用Mセッケンの6通りの方法で手指を洗浄または殺菌消毒させ,~手洗い前と手洗い後の手指の生菌数の比較を行った.~

~

高齢者介護施設における入居者の手指衛生の実態~

\href{https://www.jstage.jst.go.jp/search/global/_search/-char/ja?item=8&word=\%E4\%B8\%AD\%E6\%9D\%91+\%E6\%98\%8E\%E4\%B8\%96}{中村
明世},\,\href{https://www.jstage.jst.go.jp/search/global/_search/-char/ja?item=8&word=\%E5\%B2\%A1\%E7\%94\%B0+\%E6\%B7\%B3\%E5\%AD\%90}{岡田
淳子},\,\href{https://www.jstage.jst.go.jp/search/global/_search/-char/ja?item=8&word=\%E9\%A3\%AF\%E7\%94\%B0+\%E5\%BF\%A0\%E8\%A1\%8C}{飯田
忠行}~

本研究の目的は,
高齢者介護施設入居者の手指衛生の実態を把握することである.
入居者をケアする職員240名を対象に入居者の手指衛生の実施頻度を調査した.
入居者は活動状況別に3群に分類し,
手指衛生の方法と実施場面の違いによる手指衛生の実施頻度を調査し,
3群間で比較した.
入居者の活動状況によって介護現場で実施されている手指衛生の方法は異なり,
寝たきりのベッド上群がウェットティッシュ,
排泄をポータブルトイレで行うポータブルトイレ群はアルコール,
ひとりで移動できる見守り群は流水であった. 手指衛生の場面は,
ポータブルトイレ群と見守り群の排尿後および排便後の実施頻度が高い傾向にあった.\,~

~~

3. 本研究の目的と重要性~

現在までに発表されている実験データの中には、人が日常生活の中でさまざまな物体に手で触れた際、どの程度の微生物が皮膚表面に付着するかを測定した研究がいくつか存在している。これらの研究では、主に日常的な生活環境や医療現場など、比較的衛生状態が管理された状況を想定しているものが多い。しかし、災害発生時のように、衛生環境が急激に悪化し、安全な水の供給が途絶えるような特殊な状況を対象とした研究は、現時点ではほとんど見られないのが現状である。そこで本研究では、災害時において水道などの安全な水源の利用が困難となり、やむを得ず河川水や雨水、貯留水などの代替水源を使用せざるを得ない場合を想定し、手指にどの程度の微生物が付着・残存するのかを詳細に測定することを目的とする。特に、手洗いの方法や使用する水の種類、接触時間などの実験条件を変化させることで、微生物付着量の違いを定量的に明らかにすることを目指す。また、得られた測定データをもとに統計的・科学的な分析を行い、災害時における感染症発生のリスクを数値的に評価する。さらに、その結果を踏まえて、災害時や安全な水資源へのアクセスが制限される状況でも実践可能な効果的な手指衛生の方法や、感染症予防に有効な衛生対策を検討・提案することを本研究の最終的な目的とする。~

~

~
